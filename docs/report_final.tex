\documentclass[a4paper,12pt,notitlepage,english]{article}
\usepackage[hmargin=1.5cm,vmargin=2cm]{geometry}
\usepackage[utf8]{inputenc}
\usepackage[T1]{fontenc}
\usepackage{lmodern}
\usepackage[dvipsnames]{xcolor}
\usepackage[main=english]{babel}
\usepackage{mathtools}
\usepackage{listings}
\usepackage[pdfusetitle]{hyperref}

\lstdefinelanguage{psblm}{
  keywords={var,read,write,begin,end,if,then,else,while,do,function,return},
  comment=[s]{\{}{\}},
  string=[b]{"},
  morestring=[b]{'},
  sensitive=true,
}
\lstset{		
	basicstyle=\ttfamily\footnotesize,
  keywordstyle=\bfseries\textcolor{Orchid},
  commentstyle=\itshape\textcolor{gray},
  stringstyle=\textcolor{Green},
  tabsize=2,
  breaklines=true,
	postbreak=\mbox{\textcolor{red}{\(\hookrightarrow\)}\space},
	keepspaces=true,
  showstringspaces=false,
  numbers=left,
  numberstyle=\tiny,
  numbersep=5pt,
  stepnumber=1,
}
\renewcommand\lstlistingnamestyle{\bfseries}

\title{\textbf{PSBLM} Transpiler}
\author{\textsc{Berthoud} Tanguy \and \textsc{Liard} François \and \textsc{Msekni} Sabri}

\begin{document}
  \maketitle

  This work contains the final version of our transpiler, which transpiles our pseudo-code \emph{PSBLM} language into Python.
  The \emph{PSBLM} language grammar is described in \autoref{sec:grammar}.\\

  \noindent The code is hosted in the GitHub repository \url{https://github.com/Dragonink/ISEP_A3-Transpiler}.

  \tableofcontents\newpage

  \section{Grammar}\label{sec:grammar}

  In grammar rules, \emph{\bfseries bold italic} words are not reserved words, but literal characters: \begin{itemize}
    \item \emph{\bfseries newline} is the \verb+LINE FEED+ character
  \end{itemize}

  Please note that the following grammar is the grammar we defined at the beginning of the project.
  The implementation is equivalent but may not be exactly the same (see the \autoref{lst:syna.y}).

  \subsection{Comments}

  \(\verb+<comment>+ \longrightarrow \text{\bfseries \{} \left[\verb+^+\}\right]^* \text{\bfseries \}}\)

  \subsection{Identifiers}

  \(\verb+<letter>+ \longrightarrow \left[\text{A-Za-z}\right]\)\\
  \(\verb+<digit>+ \longrightarrow \left[\text{0-9}\right]\)\\
  \(\verb+<ident>+ \longrightarrow \verb+<letter>+ \left(\verb+<letter>+ | \verb+<digit>+ | \_\right)^*\)\\

  \noindent Identifiers (\verb+<ident>+) cannot be reserved words (in \textbf{bold} in grammar rules).

  \subsection{Variable declaration}

  \(\verb+<type>+ \longrightarrow \text{\bfseries integer} | \text{\bfseries real} | \text{\bfseries boolean} | \text{\bfseries char} | \text{\bfseries string} \)\\
  \(\verb+<declaration>+ \longrightarrow \text{\bfseries var } \verb+<ident>+\left(\text{\bfseries , } \verb+<ident>+\right)^*\text{\bfseries : } \verb+<type>+\)

  \subsection{Expressions}

  \(\verb+<number>+ \longrightarrow \verb+<digit>+^+\left(\text{\bfseries .}\verb+<digit>+^+\right)?\left(\text{\bfseries e}\left[\text{\bfseries +-}\right]?\verb+<digit>+^+\right)?\)\\
  \(\verb+<bool>+ \longrightarrow \text{\bfseries true} | \text{\bfseries false}\)\\
  \(\verb+<char>+ \longrightarrow \text{\bfseries '}\left[\verb+^+\text{'}\right]\text{\bfseries '}\)\\
  \(\verb+<string>+ \longrightarrow \text{\bfseries "}[\verb+^+"]^*\text{\bfseries "}\)\\
  \(\verb+<literal>+ \longrightarrow \verb+<number>+ | \verb+<bool>+ | \verb+<char>+ | \verb+<string>+\)\\
  \(\verb+<unop>+ \longrightarrow \text{\bfseries +} | \text{\bfseries -} | \text{\bfseries not}\)\\
  \(\verb+<biop>+ \longrightarrow \text{\bfseries +} | \text{\bfseries -} | \text{\bfseries *} | \text{\bfseries /} | \text{\bfseries div} | \text{\bfseries mod} | \text{\bfseries =} | \text{\bfseries <} | \text{\bfseries <=} | \text{\bfseries >} | \text{\bfseries >=} | \text{\bfseries <>} | \text{\bfseries and} | \text{\bfseries or}\)\\
  \(\verb+<expr>+ \longrightarrow \verb+<comment>+\verb+<expr>+ | \verb+<expr>+\verb+<comment>+\\\indent | \text{\bfseries (}\verb+<expr>+\text{\bfseries )} | \verb+<unop>+\verb+<expr>+ | \verb+<expr>+\verb+<biop>+\verb+<expr>+ | \verb+<call>+ | \verb+<ident>+ | \verb+<literal>+\)

  \subsection{Statements}

  \(\verb+<assign>+ \longrightarrow \verb+<ident>+ \text{ \bfseries := } \verb+<expr>+\)\\
  \(\verb+<read>+ \longrightarrow \text{\bfseries read(}\verb+<expr>+\text{\bfseries , }\verb+<ident>+\text{\bfseries )}\)\\
  \(\verb+<write>+ \longrightarrow \text{\bfseries write(}\verb+<expr>+\text{\bfseries , }\verb+<expr>+\text{\bfseries )}\)\\
  \(\verb+<return>+ \longrightarrow \text{\bfseries return }\verb+<expr>+\)\\
  \(\verb+<stmt>+ \longrightarrow \verb+<comment>+ | \verb+<declaration>+ | \verb+<assign>+ | \verb+<read>+ | \verb+<write>+ | \verb+<return>+ | \verb+<if>+ | \verb+<while>+ | \verb+<block>+\)\\
  \(\verb+<block>+ \longrightarrow \text{\bfseries begin \emph{newline}} \left(\verb+<stmt>+ \text{ \bfseries\emph{newline}}\right)^* \text{\bfseries end}\)\\
  \(\verb+<if>+ \longrightarrow \text{\bfseries if } \verb+<expr>+ \text{ \bfseries then } \verb+<block>+ \left(\text{\bfseries else } \verb+<block>+\right)?\)\\
  \(\verb+<while>+ \longrightarrow \text{\bfseries while } \verb+<expr>+ \text{ \bfseries do } \verb+<block>+\)

  \subsection{Functions}

  \(\verb+<fn>+ \longrightarrow \text{\bfseries function } \verb+<ident>+\text{\bfseries (}\left(\verb+<ident>+\left(\text{\bfseries , } \verb+<ident>+\right)^*\right)?\text{\bfseries )}\text{\bfseries : } \verb+<type>+ \verb+<block>+\)\\
  \(\verb+<call>+ \longrightarrow \verb+<ident>+\text{\bfseries (}\left(\verb+<expr>+\left(\text{\bfseries , } \verb+<expr>+\right)^*\right)?\text{\bfseries )}\)

  \subsection{Program structure}

  \(\verb+<program>+ \longrightarrow ((\verb+<stmt>+ | \verb+<fn>+)\text{ \bfseries\emph{newline}})^*(\verb+<stmt>+ | \verb+<fn>+)?\)\\

  \noindent\verb+<program>+ is the starting variable of our grammar.

  \section{Standard functions}\label{sec:std}

  Standard functions will be already declared identifiers.
  These are:\begin{itemize}
    \item \(|x|\) as \verb+abs(x)+
    \item \(\exp(x)\) as \verb+exp(x)+
    \item \(\log(x)\) as \verb+log(x)+
    \item \(x^p\) as \verb+pow(x,p)+
    \item \(\sqrt{x}\) as \verb+sqrt(x)+
    \item \(\lfloor x \rfloor\) as \verb+floor(x)+
    \item \(\lceil x \rceil\) as \verb+ceil(x)+
  \end{itemize}

  \section{Lexical analyser}\label{sec:lexa}

  We are using \texttt{flex} as our lexical analyser generator.
  Its role is just to recognise and pass the tokens to our syntax analyser (see \autoref{sec:syna}).

  We are including \texttt{y.tab.h}, generated when compiling our syntax analyser, so our lexical analyser uses the exact values of the tokens known to our syntax analyser.
  Moreover, it also passes the raw values of the \verb+<type>+, \verb+<ident>+ and \verb+<literal>+ variables using the special variable \texttt{yylval}.

  The implementation listing is \autoref{lst:lexa.l}.

  \section{Syntax analyser}\label{sec:syna}

  We are using \texttt{bison} as our syntax analyser generator.
  Its role is to recognise the input text against our grammar, using the tokens passed by our lexical analyser (see \autoref{sec:lexa}).

  Alongside its recognition, we programmed it to manage symbol tables as well as a parse tree.
  This tree will be used as an intermediate code representation to finally generate Python output.

  The implementation listing is \autoref{lst:syna.y}.

  \subsection{Symbol tables}

  Our syntax analyser manages a main symbol table (\texttt{symtable} in the code), but can also manage scoped symbol tables.
  To do that, it keeps track of the current scope using the \texttt{curr\_symtable} pointer.
  When the scope ends, the \texttt{curr\_symtable} pointer is updated using the \texttt{parent} pointer of the current symbol table.

  When the main symbol table is created, we manually add the standard functions (see \autoref{sec:std}) into it, so there will not be an `Undefined function' error when using them.

  We have three functions to manage our symbol tables: \texttt{search\_sym}, \texttt{add\_sym} and \texttt{scoped\_symtable}.

  \noindent\begin{minipage}{.3\linewidth}
    The \texttt{search\_sym} function searches for an existing symbol with the name \texttt{name} in the table \texttt{table}.
    If it cannot find it, it searches recursively in the parent table.
  \end{minipage}\hspace{.05\linewidth}\begin{minipage}{.65\linewidth}
    \lstinputlisting[linerange={324-336},firstnumber=324,language=C,nolol]{../src/syna.y}
  \end{minipage}

  \noindent\begin{minipage}{.3\linewidth}
    The \texttt{add\_sym} function first calls \texttt{search\_sym} to check if the symbol exists.
    If the symbol does not exist, it creates a new entry in the current table.
  \end{minipage}\hspace{.05\linewidth}\begin{minipage}{.65\linewidth}
    \lstinputlisting[linerange={337-350},firstnumber=337,language=C,nolol]{../src/syna.y}
  \end{minipage}

  \noindent\begin{minipage}{.3\linewidth}
    The \texttt{scoped\_symtable} function creates a new symbol table and sets the current table as the parent of the newly-created one.
  \end{minipage}\hspace{.05\linewidth}\begin{minipage}{.65\linewidth}
    \lstinputlisting[linerange={351-355},firstnumber=351,language=C,nolol]{../src/syna.y}
  \end{minipage}

  \subsection{Parse tree}

  Our syntax analyser manages a binary parse tree, in which each node has two pointers to their children.
  The tree root is saved in \texttt{root}.

  The main function to manage our parse tree is \texttt{new\_ast}.

  \noindent\begin{minipage}{.3\linewidth}
    Its role is only to allocate space for the node object, and set the type of the node (\texttt{token}) as well as its children.
    Linking it correctly into the tree is the responsibility of the caller.
  \end{minipage}\hspace{.05\linewidth}\begin{minipage}{.65\linewidth}
    \lstinputlisting[linerange={357-363},firstnumber=357,language=C,nolol]{../src/syna.y}
  \end{minipage}

  \section{Code generation}

  Finally after the syntax analyser returns the parse tree (see \autoref{sec:syna}), we pass this tree into our \texttt{transpile} function, which role is to output Python code.

  It will recursively traverse the parse tree, and for each node generate a valid Python code, which will be appended to the output text.
  Because of Python's indentation rules, we need to keep track of the current indentation level using the function's \texttt{indent\_lvl} parameter, which is increased or decreased along the blocks in the parse tree.

  The implementation listing is \autoref{lst:ast.c}.

  \section{Tests}

  For each test, the source \emph{PSBLM} code is the listing on the left, and the output Python code is the listing on the right.

  \subsection{Addition}

  \noindent\begin{minipage}{.475\linewidth}
    \lstinputlisting[language=psblm,nolol]{../tests/addition.psblm}
  \end{minipage}\hspace{.05\linewidth}\begin{minipage}{.475\linewidth}
    \lstinputlisting[language=Python,nolol]{../tests/addition.py}
  \end{minipage}

  \subsection{If}

  \noindent\begin{minipage}{.475\linewidth}
    \lstinputlisting[language=psblm,nolol]{../tests/if.psblm}
  \end{minipage}\hspace{.05\linewidth}\begin{minipage}{.475\linewidth}
    \lstinputlisting[language=Python,nolol]{../tests/if.py}
  \end{minipage}

  \subsection{While}

  \noindent\begin{minipage}{.475\linewidth}
    \lstinputlisting[language=psblm,nolol]{../tests/while.psblm}
  \end{minipage}\hspace{.05\linewidth}\begin{minipage}{.475\linewidth}
    \lstinputlisting[language=Python,nolol]{../tests/while.py}
  \end{minipage}

  \subsection{Function}

  \noindent\begin{minipage}{.475\linewidth}
    \lstinputlisting[language=psblm,nolol]{../tests/function.psblm}
  \end{minipage}\hspace{.05\linewidth}\begin{minipage}{.475\linewidth}
    \lstinputlisting[language=Python,nolol]{../tests/function.py}
  \end{minipage}

  \subsection{Function 2}

  \noindent\begin{minipage}{.475\linewidth}
    \lstinputlisting[language=psblm,nolol]{../tests/function2.psblm}
  \end{minipage}\hspace{.05\linewidth}\begin{minipage}{.475\linewidth}
    \lstinputlisting[language=Python,nolol]{../tests/function2.py}
  \end{minipage}

  \subsection{Standard function}

  \noindent\begin{minipage}{.475\linewidth}
    \lstinputlisting[language=psblm,nolol]{../tests/std.psblm}
  \end{minipage}\hspace{.05\linewidth}\begin{minipage}{.475\linewidth}
    \lstinputlisting[language=Python,nolol]{../tests/std.py}
  \end{minipage}

  \clearpage\pagenumbering{roman}\appendix
  \part*\lstlistlistingname\addcontentsline{toc}{part}\lstlistlistingname%

  \renewcommand\lstlistlistingname{}  
  \vspace{-1cm}\lstlistoflistings\vspace{.5cm}

  \lstinputlisting[caption={\ttfamily sym.h},label={lst:sym.h},language=C]{../src/sym.h}

  \newpage\lstinputlisting[caption={\ttfamily ast.h},label={lst:ast.h},language=C]{../src/ast.h}

  \newpage\lstinputlisting[caption={\ttfamily ast.c},label={lst:ast.c},language=C]{../src/ast.c}

  \newpage\lstinputlisting[caption={\ttfamily lexa.l},label={lst:lexa.l}]{../src/lexa.l}

  \newpage\lstinputlisting[caption={\ttfamily syna.y},label={lst:syna.y}]{../src/syna.y}
\end{document}
