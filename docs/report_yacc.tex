\documentclass[a4paper,12pt,notitlepage,english]{article}
\usepackage[hmargin=1.5cm,vmargin=2cm]{geometry}
\usepackage[utf8]{inputenc}
\usepackage[T1]{fontenc}
\usepackage{lmodern}
\usepackage{xcolor}
\usepackage[main=english]{babel}
\usepackage{mathtools}
\usepackage{listings}
\usepackage[pdfusetitle]{hyperref}

\lstset{		
	basicstyle=\ttfamily\footnotesize,
  keywordstyle=\bfseries,
  commentstyle=\itshape,
  numberstyle=\tiny,
  tabsize=2,
  breaklines=true,
	postbreak=\mbox{\textcolor{red}{\(\hookrightarrow\)}\space},
	keepspaces=true,
  showstringspaces=false,
  numbers=left,
  numbersep=5pt,
  stepnumber=1,
}

\title{\textbf{PSBLM} Transpiler Syntax Analyser}
\author{\textsc{Berthoud} Tanguy \and \textsc{Liard} François \and \textsc{Msekni} Sabri}

\begin{document}
	\maketitle

	This work contains our syntax analyser, which was built from our prototype lexical analyser.

	\section{\emph{LEX} grammar changes}

	We had to adapt our grammar to cope with the limitations of \emph{LEX} when doing our prototype lexical analyser.
	But since \emph{YACC} does not have these issues (notably with recursion), we used our grammar we first defined.

	The final grammar used (adapted to \emph{YACC}) can be seen in \hyperref[sec:lst-yacc]{\bfseries \emph{YACC} code}, starting at line 50.

	\section{Interaction between \emph{LEX} and \emph{YACC}}

	Previously \emph{LEX} was used as standalone lexical analyser.
	Now it is paired with \emph{YACC} as a token provider: \emph{YACC} asks the tokens to \emph{LEX}, and \emph{YACC} does its analysis based on the tokens returned by \emph{LEX}.

	\section{Products of \emph{YACC}}

	During our syntax analysis, we construct a symbol table (see the header file in \hyperref[sec:lst-sym]{\texttt{sym.h}}) and a parse tree (see the header file in \hyperref[sec:lst-ast]{\texttt{ast.h}}).
	These will be used to generate the output code of our transpiler.

	\section{Testing our syntax analyser}

	Our compiled syntax analyser asks an input either from \texttt{stdin} or by pipe.

	It prints an error and returns with a non-zero value if some syntax error has been encountered.
	Otherwise, it prints nothing and returns with zero.

	\newpage\appendix\part*{Listings}

	\section*{Header files}

	\subsection*{\texttt{sym.h}}\label{sec:lst-sym}

	\lstinputlisting{../src/sym.h}

	\subsection*{\texttt{ast.h}}\label{sec:lst-ast}

	\lstinputlisting{../src/ast.h}

	\section*{\emph{LEX} code}

	\lstinputlisting{../src/lexa.l}

	\section*{\emph{YACC} code}\label{sec:lst-yacc}

	\lstinputlisting{../src/syna.y}

	\newpage\section*{Tests}

	Our analyser recognizes all of the following tests as syntactically valid programs.

	\subsection*{Addition}

	\lstinputlisting{../tests/addition.psblm}

	\subsection*{If}

	\lstinputlisting{../tests/if.psblm}

	\subsection*{While}

	\lstinputlisting{../tests/while.psblm}

	\subsection*{Function}

	\lstinputlisting{../tests/function.psblm}

	\subsection*{Function 2}

	\lstinputlisting{../tests/function2.psblm}
\end{document}
